\documentclass[a4paper,11pt]{article}
\usepackage[T1]{fontenc}
\usepackage[utf8]{inputenc}
\usepackage{lmodern}

\title{Relatório}
\author{Thiago de Gouveia Nunes \\ Wilson Kazuo Mizutani}

\begin{document}

\maketitle
\tableofcontents

\clearpage
\section{Saida do Programa}
\subsection{arquivo1.txt}
\subsection{arquivo2.txt}
\section{Atribuição das Velocidades Aleatórias}
  Usamos uma distribuição uniforme de [20-80] para atribuir a velocidade para os ciclistas.
\section{Definição de Desempate}
  Para desempatar dois ciclistas, usamos a sua classificação geral, na qual nunca ocorre empate pois depende da ordem na qual cada thread
consegue a trava do ranking. Assim, se o ciclista A chegou em quinto no geral e o B em oitavo no geral, mas os dois tem 90 pontos
cada na camiseta verde, o A virá na frente do B.
\section{Outras observações}
\subsection{Delays}
  As threads do ciclistas fazem um nanosleep de 1ms a cada iteração lógica pois isso permite que o escalonador balanceie melhor a
distribuição de quotas de tempo entre elas.
\subsection{Corridas demoram mais minutos virtuais}
  Para evitar deadlocks na hora de imprimir os relatórios periódicos, tivemos que considerar que ciclistas à espera de espaço no
quilômetro seguinte também "avançassem no tempo". Portanto naturalmente acontece das corridas levarem mais minutos virtuais de tempo do
que se imaginaria ser possível. Por exemplo, num percurso de 200km, esperaria-se que o tempo máximo de corrida fosse 600min, mas com o
tempo adicional de espera dos ciclistas "com engarrafamento" esse tempo na prática acaba facilmente excedendo tal limite.
\end{document}
